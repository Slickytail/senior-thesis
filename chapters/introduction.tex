\chapter{Introduction}\label{ch:intro}

% Applications of knot theory
% DNA, 
Mathematical knots are conceptually similar to real twists of thread. But physical knots are of interest because of the friction between strands, while mathematical knots have no such interaction.
Thus the primary question of interest in mathematical knot theory is the identification, classification, and equivalence of knots. By studying the ``universe'' of knots, we study the topology of a 3-dimensional manifold, most commonly Euclidian 3-space $\R^3$. 
Knot theory also has a number of real-world applications in biology and physics, such as in the knotting of DNA, in the folding of proteins, and in fluid dynamics.

We normally define two knots as equivalent if they are smoothly isotopic.
But this is not the only possible equivalence relation, or even the only interesting one. Another that we will make use of in this paper is the existence of a cobordism --- a smooth 2-manifold having two knots as its boundary.
The existence of a cobordism is a much weaker condition than knot equivalence, so it divides the universe of knots into classes \cite{fox-milnor}.

% Applications of contact geometry
% Maybe mention briefly that contact geometry has applications in phycs, including optics and mechanics
We can also use knot theory to study spaces with more structure than the normal $\R^3$. 
In particular, by the association of a certain plane field (see Definition~\ref{defn:xi}) with $\R^3$, we are able to study \emph{contact manifolds}. Contact geometry is a rich and actively-studied field, with broad applications to physics, including geometric optics and classical mechanics. The knots of interest living in contact manifolds are \emph{Legendrian knots}, whose tangent vectors lie on the plane field.
This gives rise to the equivalence relation of Legendrian equivalence, which is \emph{strictly finer} than that of smooth equivalence.
% The number of peaks is not known.
The structure of these equivalence classes is nontrivial but somewhat understood: given a Legendrian representative $K$, additional representatives with lower Thurston-Bennequin number (defined in Subsection~\ref{subsec:invariants}) may be obtained by adding additional local twists.

% It is a result of Fuchs and Tabachnikov (\cite{fuchs-tabachnikov}) that if $K$ and $K'$ are smoothly isotopic Legendrian knots, then with sufficient stabilizations they will also be Legendrian isotopic.
% That is, stabilizations.

Moreover, each contact manifold has a canonically associated 4-manifold, equipped with a similar differential condition. Analogous to Legendrian curves in contact 3-manifolds are Lagrangian surfaces in symplectic 4-manifolds, which allow us to define a similar notion of cobordism for Legendrian knots.
This relation is in a sense finer than that of smooth cobordism, as the existence of a Lagrangian cobordism implies the existence of a smooth one, but it is not an equivalence relation on the set of Legendrian knots as it is not symmetric \cite{chantraine2015}.

Smooth cobordisms have been extensively studied, but much less is known about their Lagrangian counterparts. There are several known necessary conditions for the existence of cobordisms, some of which we will briefly mention here.
First, the existence of a Lagrangian cobordism from $K$ to $\emptyset$ is mutually exclusive with the existence of a Lagrangian cobordism from $\emptyset$ to $K$, a result of Gromov \cite{gromov}.
The existence of a Lagrangian cobordism from $K_-$ to $K_+$ also gives information about many invariants of $K_-$ and $K_+$ (see \cite{pan}, \cite{cdrg}, \cite{baldwin}). 
A particularly useful result of Chantraine gives a simple condition on the values of the classical invariants (Subsection~\ref{subsec:invariants}) of $K_-$ and $K_+$ \cite{chantraine2010}.
In addition to these obstructions, there are diagrammatically-defined sufficient conditions (see \cite{bourgeois15}, \cite{lin}, \cite{guadagni}), though it is nontrivial to use these conditions to make general positive statements about the existence of cobordisms.

% State this as a theorem!
The main result of this thesis is an infinite family of knots, $\{P_n\}$, each of which has a maximal-$\tb$ Legendrian representative $K_n$ admitting a Lagrangian cobordism from a suitably stabilized Legendrian unknot $U_n$. In fact, this allows us to construct cobordisms from stabilizations of $U_n$ to any stabilization of $K_n$. In some cases, such as $P_1$, all Legendrian representatives are stabilizations of the maximal representative.
We state the theorem here; the proof of this is the focus of Chapter~\ref{ch:pretzel}.

\begin{mythmcopy}
     Let $P_n = P(3, -3, n)$, for $n$ an integer, and define $\maxtb P_n$ to be the maximal value of the Thurston-Bennequin number over all Legendrian representatives of $P_n$. Then there exists a Legendrian representative $K$ of $P_n$, and a Legendrian unknot $U$ with $\tb K = \tb U = \maxtb P_n$, such that there is a decomposable Lagrangian concordance from $U$ to $K$.
\end{mythmcopy}


This thesis is organized as follows.
In Chapter~\ref{ch:background}, we provide background material on the theory of Legendrian knots: in Section~\ref{sec:knots}, introducing knots in the smooth context; in Section~\ref{sec:legendrian}, laying out the basics of Legendrian knots, their representations, and useful invariants; and in Section~\ref{sec:cobordisms}, defining smooth and Lagrangian cobordisms and briefly summarizing known results about Lagrangian cobordisms.
In Chapter~\ref{ch:pretzel}, we motivate and prove Theorem~\ref{thm:mine}.
In Chapter~\ref{ch:future}, we discuss other classes of knots which are promising candidates for results similar to Theorem~\ref{thm:mine} (i.e., whose maximal-$\tb$ representatives might admit Lagrangian cobordisms from the unknot).
Finally, in Appendix~\ref{ch:appendix}, we provide and explain the Mathematica code that we used to determine $\maxtb$ for the $P(3, -3, n$) family.

% Include author's name in a citation when you are actually stating the result.
